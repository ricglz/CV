%-----------------------------------------------------------------
%	SECTION TITLE
%-----------------------------------------------------------------
\cvsection{{Projects}}{{Proyectos}}
%-----------------------------------------------------------------
%	CONTENT
%-----------------------------------------------------------------
\begin{cventries}
%---------------------------------------------------------
%----------------Web Development-----------------------------
%---------------------------------------------------------

  \cventry
    {Javascript}
    {Tic-Tac-Toe-HL} % Name
    {May. 2018 - May. 2019} % Dates
    {\url{ttt-hl-react.firebaseapp.com}} % URL
    {
      \begin{cvitems} % Description(s) of experience/contributions/knowledge
        \IntlItem {{
          Created project to experiment with technologies, that's a modification of the popular Tic-Tac-Toe game
        }} {{
          Crée un proyecto para experimentar con tecnologías de desarrollo web
        }}
        \IntlItem {{
          Developed an MVP website using HTML, JS and CSS, then improved the scalability and features in another app ReactJS
        }} {{
          Desarrollé un MVP usando sólo HTML, JS y CSS, para después mejorarlo usando ReactJS
        }}
        \IntlItem {{
          Created an Open Source project that currently has 10+ contributors and 20+ forks on Github
        }} {{
          Abrí el proyecto como uno de código abierto, el cual tiene actualmente más de 10 contribuidores y más de 20 forks en Github
        }}
        \IntlItem {{
          Implemented CI pipelines for detecting code style issues and deploying Travis CI
        }} {{
          Implementé pruebas automatizadas y un sistema para hacer el deploy automáticamente usando Travis CI.
        }}
        \IntlItem {{
          Made use of serverless functions and databases using Firebase tools
        }} {{
          Usé las herramientras de Firebase para crear base de datos y funcionalidades serverless
        }}
      \end{cvitems}
    }
%---------------------------------------------------------
%----------------Mobile Development-----------------------
%---------------------------------------------------------

  \cventry
    {Swift}
    {Base-Calc} % Name
    {Feb. 2020 - Jun. 2020} % Dates
    {\url{www.github.com/ricglz/Base-Calc}} % URL
    {
      \begin{cvitems} % Description(s) of experience/contributions/knowledge
        \IntlItem {{
          Built UI components using Swift UI
        }} {{
          Desarrollé los componentes de la intefaz usando Swift UI
        }}
        \IntlItem {{
          Created unit tests for the logical functions and classes of the codebase
        }} {{
          Diseñé pruebas de unidad para todas las funciones y clases de toda la aplicación
        }}
        \IntlItem {{
          Implemented the CI workflow for testing and building the app
        }} {{
          Implementé un flujo de integración continua que probaba y compilaba la aplicación
        }}
      \end{cvitems}
    }
%---------------------------------------------------------
%----------------Data Science and AI----------------------
%---------------------------------------------------------

  \cventry
    {Python}
    {Magic Mirror} % Name
    {May. - \IntlAug 2021} % Dates
    {\url{https://github.com/ricglz/Magic-mirror}} % URL
    {
      \begin{cvitems} % Description(s) of experience/contributions/knowledge
        \IntlItem {{
          Developed a Real Time Application for Deep Fakes, using OpenCV and Pytorch
        }} {{
          Desarrollé una aplicación en tiempo real para generar Deep Fakes usando OpenCV y Pytorch
        }}
        \IntlItem {{
          Compared qualitive and quantiative performance of different techniques
        }} {{
          Hice comparaciones cualitativas y cuantitaticas de diferente técnicas para Deep Fakes
        }}
        \IntlItem {{
          Wrote a Masters dissertation based on the discoveries of this project
        }} {{
          Escribí una tésis para maestría basda en los aprendizajes de este proyecto
        }}
      \end{cvitems}
    }
%---------------------------------------------------------
  \cventry
    {Python}
    {Stock Analyzer} % Name
    {Oct. 2020} % Dates
    {\url{www.github.com/ricglz/stock-analyzer}} % URL
    {
      \begin{cvitems} % Description(s) of experience/contributions/knowledge
        \IntlItem {{
          Used pandas for dataframe management of stock data
        }} {{
          Usé pandas para administrar los datos de acciones
        }}
        \IntlItem {{
          Pulled stocks historical data using Yahoo Finance API
        }} {{
          Utilicé los datos de las acciones usando la API de Yahoo Finance
        }}
        \IntlItem {{
          Calculated technical indicators using 3rd-party library
        }} {{
          Calculé los indicadores técnicos cuantitativos usando librerías auxiliares
        }}
      \end{cvitems}
    }
%---------------------------------------------------------

  \cventry
    {Python}
    {Lander experiment}
    {\IntlDec 2020}
    {\url{www.github.com/ricglz/lander-experiment}}
    {
      \begin{cvitems}
        \IntlItem {{
          Cleaned prior code of a game engine for a lander game
        }} {{
          Limpié el código anterior para el game engine del juego
        }}
        \IntlItem {{
          Added an API to be able to control the player using AI
        }} {{
          Añadí un API para poder controlar el player siendo un AI
        }}
        \IntlItem {{
          Implemented Supervised and Reinforcement learning algorithms for the bot
        }} {{
          Implementé algoritmos de aprendizaje supervisado y reforzado para este juego
        }}
      \end{cvitems}
    }
%---------------------------------------------------------
%----------------School Projects--------------------------
%---------------------------------------------------------

  \cventry
    {Javascript; Ruby}
    {Annotate-it}
    {May, 2020}
    {\url{https://annotate-it-25533.firebaseapp.com}}
    {
      \begin{cvitems}
        \IntlItem {{
          Created react-typescript application for managing with crud notes and folders
        }}{{
          Crée un sitio web usando ReactJS y Typescript para organizar notas, folders y categorías
        }}
        \IntlItem {{
          Designed UI for the application and rendering with markdown notation
        }}{{
          Diseñé una interfaz de usuario, capaz de aceptar notación tipo markdown para las notas
        }}
        \IntlItem {{
          Developed GraphQL API using ruby on rails with authentication
        }}{{
          Desarrollé una API de GraphQL usando Ruby on Rails con autentificación
        }}
      \end{cvitems}
    }
%---------------------------------------------------------
%----------------Other------------------------------------
%---------------------------------------------------------

  % \cventry
  %   {Java}
  %   {Discoords}
  %   {\IntlJan 2021}
  %   {\url{www.github.com/ricglz/discoords}}
  %   {
  %     \begin{cvitems}
  %       \item {Developed a Minecraft plugin to help specific needs that my community had}
  %       \item {Learned the Bukkit and Discord APIs for the development}
  %       \item {Wrote documentation for new users to how to install the plugin}
  %       \item {Released newer versions with improvements or fixes through Github}
  %     \end{cvitems}
  %   }
\end{cventries}
